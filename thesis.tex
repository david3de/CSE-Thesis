%% uctest.tex 11/3/94
%% Copyright (C) 1988-2004 Daniel Gildea, BBF, Ethan Munson.
%
% This work may be distributed and/or modified under the
% conditions of the LaTeX Project Public License, either version 1.3
% of this license or (at your option) any later version.
% The latest version of this license is in
%   http://www.latex-project.org/lppl.txt
% and version 1.3 or later is part of all distributions of LaTeX
% version 2003/12/01 or later.
%
% This work has the LPPL maintenance status "maintained".
% 
% The Current Maintainer of this work is Daniel Gildea.

\documentclass[12pt]{ucthesis}
\usepackage{graphicx}
\graphicspath{ {./img} }
\def\dsp{\def\baselinestretch{2.0}\large\normalsize}
\dsp
\begin{document}

% Declarations for Front Matter

\title{Children and IoT Devices}
\author{Yifu Lang}
\degreeyear{2021}
\degreemonth{June}
\degree{MASTER OF SCIENCE}
\chair{Professor Alvaro Cardenas}
\committeememberone{Professor Darrell Long}
\committeemembertwo{Professor David Harrison}
\numberofmembers{3} %% (including chair) possible: 3, 4, 5, 6
\deanlineone{Quentin Williams}
\deanlinetwo{Acting Vice Provost and Dean of Graduate Studies}
\deanlinethree{}
\field{Computer Science and Engineering}
\campus{Santa Cruz}

\begin{frontmatter}

\maketitle
\copyrightpage

\tableofcontents
\listoffigures
\listoftables

% ABSTRACT %%%%%%%%%%%%%%%%%%%%%%%%%%%%%%%%%%%%%%%%%%%%%%%%%%%%%%%%%%%%%%%%%%%%%%%%%%%%
\begin{abstract}
With the recent increase in popularity of the Internet of Things, many companies quickly developed new devices using this technology to stake a claim in a blooming industry. While taking advantage of this technology has yielded many benefits and conveniences in the past few years, the glaring issue of security has not been properly addressed. We have seen many recent attacks on children through IoT devices, and while security mechanisms do exist and are available for use, general users without a technical background may not see the risks as something worth putting effort into mitigating. 

In this thesis, we perform a network analysis on some of the existing products available on the market and collect survey data from parents and their understanding of IoT technologies. We then address the issues in IoT security and provide a solution to persuade more users to be wary of the risks at hand when ignoring proper attention to the protection of their data and privacy. 
\end{abstract}

% DEDICATION %%%%%%%%%%%%%%%%%%%%%%%%%%%%%%%%%%%%%%%%%%%%%%%%%%%%%%%%%%%%%%%%%%%%%%%%%%
\begin{dedication}
\null\vfil
{\large
\begin{center}
To my friends, family,\\\vspace{12pt}
and all who have been there for me through rough times.\\\vspace{12pt}
\end{center}}
\vfil\null
\end{dedication}

% ACKNOWLEDGEMENTS %%%%%%%%%%%%%%%%%%%%%%%%%%%%%%%%%%%%%%%%%%%%%%%%%%%%%%%%%%%%%%%%%%%%
\begin{acknowledgements}
I begin by giving my most sincere gratitude for my thesis advisor, Professor Alvaro Cardenas, as someone who has accompanied me throughout the entire research process and always provided further readings and resources for this thesis.

I would also like to thank Professor Su-hua Wang for guiding me through the process of survey and data collection. Additionally, I extend my gratitude to Elizabeth Goldman, Binaisha Datoor, Adam Ernst, and the entire Baby Lab at UCSC for helping me tread unfamiliar territories in the realm of research.

I want to thank Professor Darrell Long and Professor David Harrison as reading committee chairs for this thesis, and as some of the most influential professors I have had an opportunity to work with in my time at UCSC.

Finally, I want to acknowledge my SASE family, who have inspired me to go above and beyond in my career and have always been a shoulder to lean on when times are tough.

\end{acknowledgements}

\end{frontmatter}

% INTRODUCTION %%%%%%%%%%%%%%%%%%%%%%%%%%%%%%%%%%%%%%%%%%%%%%%%%%%%%%%%%%%%%%%%%%%%%%%%
\chapter{Introduction}
In recent years, society has found an astonishing number of ways to integrate the Internet into day-to-day devices. From smartphones to wearable technology, all devices that are connected to the Internet are encapsulated under the Internet of Things (IoT) \cite{gubbi:iot}. In recent history, modern companies have been finding different ways to incorporate this realm into households; for example, Google and Amazon have extensive development in smart home devices and toy companies have been developing smart toys for children. Overall, the expansion of IoT has enabled massive growth in recent years. Reports have shown that the number of existing Internet-connected devices is projected to grow to 14.7 billion in 2023, 48\% of which accounts for connected home applications, such as smart home devices and security cameras. \cite{cisco}.

However, as a newly blooming technology, the security aspects of these devices are still mostly unexplored. Reports of data leakage and account breaches \cite{wp:camera} for smart devices are rather common as a result of the lack of concern for security on Internet-connected devices; evidently, the importance of implementing security mechanisms into smart devices increases as adversarial parties find new ways to target and attack IoT devices.

Furthermore, with recent expansion of devices for children, parents may be putting their children's at risk. Developers, on the other hand, must also be aware of the additional security measures necessary to satisfy laws such as the Children's Online Privacy Protection Act (COPPA) \cite{reyes:coppa}. As a younger generation becomes more accustommed to an increasingly technological world, we must recognize the risks that children are exposed to when using an IoT device. 

While other works have delved into topics involving smart devices in both security and marketing, we are concerned with a parent's mental model of how IoT and cloud services work and what they are concerned about when their children are exposed to these technologies. We then want to analyze on-the-market products with their cloud architectures and develop a threat model of possible attack vectors. We aim to compare the two models and recommend implementable changes and improvements upon existing features.

In the remainder of this thesis, Chapter \ref{ch:background} begins with exploring the backgrounds of IoT technologies and an analysis of end-user security. Chapter \ref{ch:methodology} explains the methodology of our research process and provides details on our data collection and network analysis. Chapter \ref{ch:mental} and \ref{ch:taxonomy} go into detail on our survey data and technical data, respectively. Finally, chapter \ref{ch:conclusion} concludes our research by recommending security mechanisms to better protect a user's data based on our findings through survey data and network analysis.

% BACKGROUND %%%%%%%%%%%%%%%%%%%%%%%%%%%%%%%%%%%%%%%%%%%%%%%%%%%%%%%%%%%%%%%%%%%%%%%%%%
\chapter{Background}
\label{ch:background}
We begin our investigation by reviewing existing issues found in IoT devices and children's interactions on the Internet.

\section{End-User Perceptions}
While most Internet users understand its inherent risks, it can often be difficult for less experienced users to follow recommended security practices in order to keep their data safe; even the most sophisticated security mechanisms are useless if a user refuses to employ them. Oftentimes, reinforcement learning helps individuals develop habits and pattern recognition, but a lack of concrete results deters these practices. For example, positive outcomes in good cybersecurity awareness include not having one's data breached and not being the target of an attack. Negative outcomes in bad cybersecurity practices include exposure of data and loss of resources, which may not occur at all. Research has shown that due to a lack of tangible results in both positive and negative reinforcement in cybersecurity practices, it can often be hard to motivate their integration \cite{west:psychology}.

We aim to model, in general, a parent's perception of modern IoT devices. Previous research conducted by Zeng et al. \cite{zeng:enduser} has shown that smart home users with a technical background tended to better understand cloud infrastructure and how data is collected from a device and transported over a network. Additionally, these users have also shown more concern about the privacy of the data collected by these smart home devices. Inversely, users with less experience with technology tend to show less concern for their data and show little to no understanding of their device's network infrastructure. 

However, a separate study by McReynolds et al. \cite{mcreynolds:toysthatlisten} that parents show more concern for data privacy when their children are involved. When asked questions regarding smart toy and smart device privacy, many parents showed concern for data collection and parental controls, while some of the older children inteviewed expressed concern for a lack of privacy from their own parents.

Our investigation may show that parental care plays a role in cybersecurity awareness, which affects a parent's likelihood to purchase and use IoT devices. For our research, our goal is to build a generalizable mental model that encapsulates parents' privacy concerns for their children, understanding of cloud networks, and willingness to allow their children to use a smart device. 

\section{IoT Security Standards}
Although some companies 

\section{Protecting Children}
Under United States law, children's data is heavily protected under the Children's Online Privacy Protection Act (COPPA). When providing a service or product that is connected to the Internet to children under the age 13, developers are legally required to obtain parental permission and inform users of data collection. Data collection must be kept to a bare minimum for a product to function and collected data must be available for review and deletion.

While meeting COPPA standards    \cite{reyes:coppa}

% write about this
\cite{le:skillbot}


% METHODOLOGY %%%%%%%%%%%%%%%%%%%%%%%%%%%%%%%%%%%%%%%%%%%%%%%%%%%%%%%%%%%%%%%%%%%%%%%%%
\chapter{Methodology}
\label{ch:methodology}
Our research delves into two separate fields. We explore both end-user psychology through survey data as well as the technical aspects of existing IoT products that are currently available on the market.

\section{End-User Mental Model}
To build a generalizable mental model on parents and IoT devices, 

\section{Device Taxonomy}
In the technical aspects of our research, we look into two smart devices designed for children that are available on the market. 

\subsection{Network Analysis}
We inquire the security standards of these devices to 


\begin{figure}
    \includegraphics[width=\textwidth]{experiment.jpg}
    \caption{A high level overview of our experiment setup.}
\end{figure}

% MENTAL MODEL %%%%%%%%%%%%%%%%%%%%%%%%%%%%%%%%%%%%%%%%%%%%%%%%%%%%%%%%%%%%%%%%%%%%%%%%
\chapter{End User Mental Model}
\label{ch:mental}


% TAXONOMY %%%%%%%%%%%%%%%%%%%%%%%%%%%%%%%%%%%%%%%%%%%%%%%%%%%%%%%%%%%%%%%%%%%%%%%%%%%%
\chapter{Device Taxonomy}
\label{ch:taxonomy}


% CONCLUSION %%%%%%%%%%%%%%%%%%%%%%%%%%%%%%%%%%%%%%%%%%%%%%%%%%%%%%%%%%%%%%%%%%%%%%%%%%
\chapter{Conclusion}
\label{ch:conclusion}


% BIBLIOGRAPHY %%%%%%%%%%%%%%%%%%%%%%%%%%%%%%%%%%%%%%%%%%%%%%%%%%%%%%%%%%%%%%%%%%%%%%%%
\nocite{*}
\bibliographystyle{plain}
\bibliography{uctest}


% APPENDIX %%%%%%%%%%%%%%%%%%%%%%%%%%%%%%%%%%%%%%%%%%%%%%%%%%%%%%%%%%%%%%%%%%%%%%%%%%%%
\appendix
\chapter{Survey Questions}
The following section contains the questions we ask in our survey. Some topics include experience with technology, perceptions of our subject devices, data privacy, and demographics.

\begin{itemize}
    \item Smart device related questions
    \begin{itemize}
        \item If you own a smart home device, how long have you been using it?
        \item Does your child have access to a smart home device?
        \item If you own any other smart devices (e.g. smart watch, smart TV, etc.), how long have you been using it?
        \item Does your child use smart devices other than smart home devices?
        \item If you do \textbf{not} own any smart devices, would you consider buying one? Will your child have access to it? Why or why not?
    \end{itemize}
    \item Subject device questions
    \begin{itemize}
        \item Have you heard about this toy/device? If yes, please explain where.
        \item What do you like about the toy/device?
        \item What do you dislike about the toy/device?
        \item Would you consider purchasing the toy/device? If so, why? If not, what kind of improvements would you like to see be made for you to reconsider?
        \item What do you expect to be in the toy/device's privacy policy?
    \end{itemize}
    \item Scale questions (scale on Disagree-Agree)
    \begin{itemize}
        \item I have concerns for how my data is collected and stored on the Internet.
        \item I find it important that collected data is kept private.
        \item I thoroughly research into Internet-connected devices before making a purchase.
        \item I am familiar with the Children’s Online Privacy Protection Act (COPPA).
    \end{itemize}
    \item Data privacy questions
    \begin{itemize}
        \item Please describe your understanding of how Internet-connected devices collect data. 
        \item Please describe your understanding of where and how this data is stored.
        \item Please describe any concerns for how your data is stored on the Internet.
        \item What practices does your family use to keep your data safe?
        \item Have you had any security concerns with devices connected to the Internet?
        \item What concerns do you have about the dialogue between your child and the devices presented in this survey?
    \end{itemize}
    \item Demographics questions
    \begin{itemize}
        \item Age 
        \item Age(s) of children 6 years old or younger
        \item Highest level of education
        \item Occupation
        \item Family household income
        \item Experience with technology
    \end{itemize}
  \end{itemize}
\end{document}
